% Format: If-Then structure
\usepackage{ifthen}

% Format: Code beautify
\usepackage{listings}

% Old Format: code highlight
% \lstset{xleftmargin=1em,xrightmargin=1em}
% \lstset{
%     framexleftmargin=0.5em,
%     framexrightmargin=1em,
%     framextopmargin=0em,
%     basicstyle=\footnotesize,
%     framexbottommargin=1em
% }
% \lstset{
%     backgroundcolor=\color{grey},
%     commentstyle=\color{darkgreen},
%     keywordstyle=\color{blue},
%     %     caption=\lstname,
%     basicstyle=\ttfamily\footnotesize,
%     breaklines=true,
%     columns=flexible,
%     mathescape=fause,
% %    backgroundcolor=\color{backcolour},   
% %    commentstyle=\color{codegreen},
% %    keywordstyle=\color{magenta},
% %    numberstyle=\tiny\color{codegray},
% %    stringstyle=\color{codepurple},
% %    basicstyle=\footnotesize,
% %    breakatwhitespace=false,         
% %    breaklines=true,                 
% %    captionpos=b,                    
% %    keepspaces=true,                 
% %    numbers=left,                    
% %    numbersep=5pt,                  
% %    showspaces=false,                
% %    showstringspaces=false,
% %    showtabs=false,                  
%     tabsize=4,
%     numbers=left,
%     stepnumber=1,
%     numberstyle=\small,
%     numbersep=1em
% }
% \lstloadlanguages{
%     C,
%     C++,
%     Java,
%     Matlab,
%     Python,
%     Bash,
%     Mathematica
% }
\renewcommand{\lstlistingname}{代码}
\usepackage{caption}
\DeclareCaptionFont{white}{\color{white}}
\DeclareCaptionFormat{listing}{\colorbox[cmyk]{0.43, 0.35, 0.35,0.01}{\parbox{\textwidth}{\hspace{15pt}#1#2#3}}}
\captionsetup[lstlisting]{format=listing,labelfont=white,textfont=white, singlelinecheck=false, margin=0pt, font={bf,footnotesize}}
\usepackage{listings}
\lstset{
    basicstyle=\footnotesize\ttfamily, % Standardschrift
    %numbers=left,               % Ort der Zeilennummern
    numberstyle=\tiny,          % Stil der Zeilennummern
    %stepnumber=2,               % Abstand zwischen den Zeilennummern
    numbersep=5pt,              % Abstand der Nummern zum Text
    tabsize=4,                  % Groesse von Tabs
    extendedchars=true,         %
    breaklines=true,            % Zeilen werden Umgebrochen
    keywordstyle=\color{red},
    frame=b,         
    escapeinside=``,
    keywordstyle=[1]\textbf,    % Stil der Keywords
    keywordstyle=[2]\textbf,    %
    keywordstyle=[3]\textbf,    %
    keywordstyle=[4]\textbf,    %\sqrt{\sqrt{}} 
    %stringstyle=\color{white}\ttfamily, % Farbe der String
    showspaces=false,           % Leerzeichen anzeigen ?
    showtabs=false,             % Tabs anzeigen ?
    xleftmargin=17pt,
    framexleftmargin=17pt,
    framexrightmargin=5pt,
    framexbottommargin=4pt,
    %backgroundcolor=\color{lightgray},
    commentstyle=\color{green}, % comment color
    keywordstyle=\color{blue},  % keyword color
    stringstyle=\color{red},    % string color
    showstringspaces=false      % Leerzeichen in Strings anzeigen ?        
}
\lstloadlanguages{% Check Dokumentation for further languages ...
    %[Visual]Basic
    %Pascal
    C,
    C++,
    Python,
    Java
}
 

\lstset{
  language = verilog,
  numbers=left, 
  frame=shadowbox,
  basicstyle=\ttfamily,
  columns=fullflexible,%可以自动换行
  linewidth=1\linewidth, %设置代码块与行同宽
  breaklines=true,%在单词边界处换行。
  showstringspaces=false, %去掉空格时产生的下划的空格标志, 设置为true则出现
  breakatwhitespace=ture,%可以在空格处换行
  escapechar=`%设置转义字符为反引号
}
